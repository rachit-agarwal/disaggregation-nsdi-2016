\vspace{-0.1in}
\section{Introduction}
\vspace{-0.05in}
\label{sec:intro}
Existing datacenters are based on a server-centric architecture, where a small amount of the resources needed for computing tasks (CPU, memory, storage, interconnects) are tightly coupled within a single server. While the server-centric architecture has been a tremendous success, recent industry trends suggest a paradigm shift --- a {\em disaggregated} data center (\dis) architecture that comprises of a pool of resources, each built as a standalone resource blade, interconnected via a datacenter-wide network fabric. Examples abound already --- Intel RSA~\cite{rsa}, HP ``the machine''~\cite{hptm}, Facebook's disaggregated rack~\cite{fdr}, and SeaMicro~\cite{seamicro}, as well as research prototypes like Firebox~\cite{firebox} and soNUMA~\cite{sonuma}.

Resource disaggregation is beneficial along several dimensions. First, disaggregation makes computing hardware more modular, enabling the technology for each individual resource to evolve independently. Indeed, as new technologies evolve or need for specialized computing needs arise, physically decoupling the resources alleviates the burdensome process of integration, server factor form planning, and motherboard designs. Second, resource disaggregation also provides a fine-grained control over provisioning, sharing and efficiently utilizing individual resources. Finally, resource disaggregation allows overcoming the technology barriers (imbalance between memory and CPU capacity, power dissipation issues, etc), potentially enabling new technological advances. 

While resource disaggregation offers aforementioned benefits, it delegates a number of scalability challenges to the underlying network fabric. Indeed, the low-latency high-bandwidth inter-resource communication that used to be contained within a server is now spread across the network fabric. This not only increases the load on the network but makes low latency communication critical. The network will, thus, be one of the key enabling or blocking factors to resource disaggregation. 

Unfortunately, the network support required for resource disaggregation is, at best, poorly understood. The existing deployments of disaggregated datacenters are either small scale or proprietary, with few details available publicly~\cite{rsa, hptm, fdr, seamicro}.

This paper takes a step in building an understanding of the requirements and challenges in enabling network support for \dis. Our approach is workload-driven; we use five common, yet diverse, workloads including batch processing jobs from Hadoop and Spark, point queries from Memcached~\cite{memcached} and ElasticSearch~\cite{elastic} and streaming applications from Storm~\cite{storm}. Using these workloads, we aim to answer three questions through a combination of emulations and simulation: 

\begin{itemize}[leftmargin=*]
	\itemsep0em
		\item What support will be needed from the network, in terms of latency and bandwidth, to maintain the application-level performance within $10\%$ of server-centric architecture? \rc{$\gets$ a lot of people asked yesterday: why $10\%$?}
	\item How will the network traffic change in DDC? What are the important design parameters that impact traffic patterns?
    \item Can existing transport protocols meet the above requirements? 
\end{itemize}

\noindent
To date, there is no consensus on the granularity at which resource disaggregation will happen --- at the rack-scale, pod-scale, or an extreme of datacenter scale. Moreover, as briefly discussed above, resource disaggregation enables flexibility in choice of provisioning and sharing of resources adding to the degrees of freedom in design of \dis architecture. Given that our focus is on understanding the network support for \dis (rather than proposing a \dis architecture), we consider the new degrees of freedom --- scale of disaggregation, CPU-memory disaggregation, data placement and access, etc. --- as design parameters that may impact our study. Our key findings are:

\begin{itemize}[leftmargin=*]
	\itemsep0em
		\item To maintain the application-level performance within $10\%$ of current server-centric architecture, \dis will require a full-bisection bandwidth network with $40$Gbps bandwidth capacity, and an end-to-end latency of $5\mu$s, ignoring the optimizations that disaggregation enables. These requirements highlight the challenge: while $40$Gbps (or even $100$Gbps) is feasible with existing technology, the key to resource disaggregation is to achieve low end-to-end latency between resource blades. \rqc{Independent of the design parameters in DDC;} \rc{$\gets$ removed this sentence; it is not entirely true that this result is independent of design parameters; eg, this result depends on local cache size}
	    \item As expected, DDC traffic volume increases; however, in comparison to existing datacenter traffic studies~\cite{srikanth, theo}, we observe: (a) relatively more homogeneous spatial and temporal distribution of traffic; (b) concentration of traffic within a smaller range of flow sizes; and (c) more homogeneous traffic volume distribution between short flows and long flows. While the design parameters significantly impact DDC traffic characteristics, the high-level observations above suggest that many common tenets that have guided datacenter design to date no longer hold. 
        \item Short memory-bound flows dominating the \dis traffic, combined with a more homogeneous spatial and temporal traffic distribution present favorable characteristics for existing transport protocols~\cite{pfabric} to meet \dis requirements. However, depending on the design parameters, existing protocols may be subjected to new challenges namely, robustness across workloads and application-level policies on flow prioritization (\S\ref{sec:existing}). \rc{$\gets$ requires better arguments} \rqc{room for improvement?}
\end{itemize}

\noindent
Given the relatively forward-looking nature of our study, several caveats apply to our observations. First, our results use workloads from applications designed for server-centric architectures. Our study is not comprehensive in existing workloads; besides, both the applications and the workloads may evolve alongside \dis architectures thus impacting  our results. Second, the focus of our study is on network support for \dis; to that end, we ignore several systems related challenges. It may very well turn out that the latter have a more profound impact on \dis architectures and applications. In this sense, one might view our study as an exploration of whether it is possible to avoid the network from becoming the bottleneck in \dis architectures. Finally, many aspects of the overall \dis context we consider do not exist yet. Consequently, we must make assumptions (\eg, resource blade organization, data layout, etc.). We make what we believe are sensible choices, state these choices explicitly and in many cases, explore the impact of the choices made on our results. However, our results are dependent on these choices and more experience is needed to confirm their validity.

\cut{
\noindent
\sr{Add a list of caveats/disclaimers as follows\ldots}
\begin{itemize} 
\item our results are based on the workloads we study; not comprehensive
\item we focus on net design, ignore many systems questions; may still well turn out that the latter matters more; in this sense one might view our study as seeing whether the network can `get out of the way’ \cite{};
\item because our study is forward looking, many aspects of the overall context we’re considering don’t exist yet so we must make assumptions - e.g., data layout, how resources blades are organized, etc.  We 
\end{itemize} }
\section{Related Work and Discussion}
\label{sec:discussion}

\subsection{Related Work}
\paragraphb{Research prototypes} As mentioned earlier, there is growing interest in rack scale disaggregation in both industry and research.
Early industry rack scale prototypes include AMD SeaMicro~\cite{seamicro}, Intel RSA~\cite{rsa}, HP's The Machine~\cite{hptm}, and Facebook's Disaggregated Rack~\cite{fdr}.
We describe HP's The Machine as an example. 
The Machine uses optical network to interconnect CPUs to 4 massive pools of memory, each with 1TB RAM.
Each node is comprised of a 64-bit ARM Sillicon on Chip (SoC), and a FPGA for communication. 
Similar to our design, each node in The Machine contains 256GB of local memory as an extra layer of cache.
While these systems disaggregate compute resources at rack scale, our paper, as shown in \S \ref{sec:existing}, explores the feasibility of scaling out disaggregation to the entire datacenter.


There are several other research prototypes such as soNUMA~\cite{sonuma}, Catapult~\cite{catapult}, and Firebox~\cite{firebox}.
Firebox~\cite{firebox}, for example, is a disaggregated prototype using high-radix silicon photonic switches to connect SoCs and memory modules numbering in the thousands.
These projects do not focus on the network design questions that we address; our efforts are complementary and may be useful in informing the design of such systems.


\paragraphb{Specialized hardware for disaggregation}
Recent works from Lim et al.~\cite{ddcHwDesign1, ddcHwDesign2} show the trend of reaching the ``memory capacity wall'' because of the growing peak compute-to-memory ratio. 
They thus propose to use remote memory blade for memory capacity expansion. 
There results demonstrate that memory disaggregation provides 10x performance improvement in memory constraint environments, which motivates our work.
~\cite{ddcHwDesign3} uses ASIC based interconnect fabric to build virtualized I/O system for better resource sharing. 
However, these interconnects designed for their specific context and do not discuss network support for disaggregation more generally nor consider the possibility of leveraging known datacenter network technologies to enable disaggregation.


\paragraphb{Distributed Shared Memory (DSM)}
RDMA allows remote memory access without OS involvement and hence enables efficient memory sharing between servers in datacenter.
FaRM~\cite{farm} does lock-free reads over RDMA and enables single machine transaction using function shipping. 
Grappa~\cite{grappa} is a software distributed memory system for data-intensive applications.
These systems demonstrate the benefits of memory sharing and show that RDMA could improve the efficiency of memory sharing.

\paragraphb{Disaggregated Datacenter Network}


R2C2~\cite{r2c2} designs a torus like topology to increase the path density in its rack scale network.
By leveraging the relatively small scale of the network, R2C2 uses a broadcasting protocol to ensure all nodes in the network are aware of all the flows in the network.
We believe broadcasting is cost prohibitive in datacenter scale distribution, and hence R2C2 is only suitable for rack scale disaggregation.

In the prior work ~\cite{hotnets}, we studied the impact of remote memory access latency to application performance in a single machine.
We extend our study by running distributed applications on 5 EC2 nodes.
Our result shows that low latency network would be the enabling factor of disaggregation, and the application level performance is not very sensitive to the scale of disaggregation.






\subsection{Limitations}
One limitation of our results is our application-agnostic measurement approach. While this models disaggregation at the operating system level, it misses any potential application-specific considerations such as the interdependency of flows and its effect on performance. As a result we did not model potential application-specific optimizations such as custom data placement strategies or prioritizing flows based on the access dependency graph. We also did not study failure modes or running mixes of applications together. As a result our results focus on demonstrating the feasibility of disaggregation rather than showcasing novel designs. Future work in this regard may focus on improving resource packing or data access scheduling mechanisms.


\sr{TBD: if we have time, let's consider adding the Spark results here presented as we went searching for a case where disaggregation fails. Found it in Spark. Explain why. And say that some apps will have to be rewritten. More generally, the question of what's the right  programming model is wide open.}





%\subsection{Looking Ahead}

%In this section, we briefly discuss some of the research questions that disaggregation raises on three fronts: 
%i) approaches to building low-latency networks, ii) network architecture, and iii) systems architecture. Each of these merits a paper in itself; as such, what follows is more an enumeration than an in-depth discussion of potential issues.

%\subsubsection{Realizing Low Latency Networks}
%As demonstrated in \S\ref{sec:requirements}, building low latency networks (with round-trip times under of 10 us \rc{change number?}) will be critical for large scale disaggregation. 
%Fortunately, this is a topic of that has been receiving a great deal of attention in recent research and, coincidentally, a recent paper~\cite{lowlatency} argues the feasibility of such low latency in datacenter networks in the near future. 
%The authors cite the growing prevalence of cut-through switches and vendor plans for
%tighter integration of IO capabilities into the CPU as key enabling factors.
%In addition to the hardware trends, we believe there are many opportunities to further reduce network latency including improved protocol designs~\cite{pfabric, phost} and all-optical switches with no buffering. 
%The effectiveness and suitability of these approaches for disaggregation is a topic for future work.
%An orthogonal approach to reducing latency is to try and reduce the network distance between the resources allocated to a job, in much the same way that map-reduce schedulers today aim for data locality in scheduling tasks.
%Future research should study how to best distribute resource blades across racks and the design of scheduler optimizations for low latency.
%Finally, an important goal should be to achieve latencies that are not just low but also deterministic, since high variability will lead to unpredictable application performance. 
%An intriguing possibility here is the use of TDMA-based network architectures as proposed in recent work by Vattikonda et al~\cite{tdma}.


%\subsubsection{Network Architecture}

%We can probably build networks for disaggregated datacenters using existing networking technologies such as Ethernet, InfiniBand, or PCIe \rc{is this true????}. 
%An interesting research question - if only to understand what change might be desirable - is to ask what the ideal network architecture in support of disaggregation might look like.

%It is worth noting that disaggregation effectively blurs the lines between what used to be separate intra- and inter-server networks. E.g., in Figure \ref{fig:dc} we see that today’s server architecture includes networks for communication between CPUs (e.g., Intel QuickPatch Interconnect or AMD HyperTransport protocols), between CPUs and memory (DDR3) and to peripheral devices (e.g.,based on the PCIe protocol). 
%Traditionally, these intra- and inter-server technologies have evolved very differently. 
%Basic concepts such as variable-sized packets and best-effort service are common in inter-server networks but not so in intra-server links/networks. 
%The network in a disaggregated datacenter combines aspects of both; it is resource-centric (like intra-server networks today) but is less tightly integrated with the endpoints and must operate at scale (like existing inter-server networks) and hence picking new network abstractions should be done carefully.
%A starting point might be to ask whether packets are the right abstraction. Since both existing intra- (except for CPU-to-memory DDR3) and inter-server link protocols today use packet-like switched technologies, we believe packets remain the right abstraction. 
%An open question however is whether we would be better served with solutions that allow us to amortize per-packet processing overheads (for reduced latency) such as larger MTUs or a “packet bursts” abstraction. \rc{don't we address this? not really an open question...}
%A second question regards communication reliability. Clearly, the resource endpoints must see an end-to-end abstraction of reliable communication; however, it is not clear whether we need reliability at the level of individual network links (as found in intra-server link
%technologies and some inter-server links such as InfiniBand) or whether end-to-end retransmissions (as used with Ethernet networks) will suffice. 
%Our conjecture is that end-to-end retransmissions should be adequate given the low RTTs we envisage, however this is an important question that warrants more rigorous exploration.
%Another related question is whether we need support for bandwidth reservations, or fair resource sharing mechanisms, or whether pure statistical multiplexing with end-to-end congestion control will suffice. \rc{this is answered in section 5}
%There are many calls for reservations and fairness~\cite{faircloud,elasticswitch,seawall} even in existing datacenter networks – if the case for these mechanisms in existing datacenters proves compelling then it is likely to be only stronger in a disaggregated datacenter (since the network’s impact on application performance is only greater). \rc{this is answered in section 5}
%We leave exploring the case for such mechanisms and the form of necessary solutions to future work.\rc{this is answered in section 5}


%\subsubsection{Systems Architecture}
%The cost of hardware and its maintenance has been the most powerful driving force of datacenter evolution, such as migration from powerful mainframes to commodity servers~\cite{casefornow}. 
%We believe that a disaggregated datacenter will be cheaper than the server-centric architecture, because i) the operator has finer-grained control over provisioning decisions, ii) disaggregated resources can simplify management complexity, and iii) the unified network cuts out a layer of integration (in lieu of the PCIe-Ethernet-PCIe traverse in current server-to-server communication). 
%In some sense, disaggregation is an extreme extrapolation of the streamlining and customization efforts that have been made by the biggest datacenters~\cite{opencompute,googlecluster}. Although the cost reduction from disaggregation is hard to quantify at this point, we suspect that cost savings might turn out to be one of the strongest motivations for disaggregated datacenters.

%In this paper, we tried to answer if we can disaggregate resources across an entire datacenter. While we are positive that disaggregation is feasible and quite likely going to happen as evidenced by our experiments and the current trends, one question still remains: 
%what is the right scale for disaggregation? Resources can be disaggregated at many different levels, such as server, rack, pod, datacenter, or something else. 
%The answer will depend on the level of savings due to disaggregation and the networking costs, and we will need to quantify this trade-off. \rc{section 5 datacenter vs rack scale}
%While the VM-as-a-unit assumption made in \S\ref{ssec:system} is a good starting point as it can readily utilize existing software infrastructures, we speculate that disaggregation may enable a more intuitive abstraction for modern datacenter applications. 
%Jobs can be most naturally described in terms of their resource requirements - e.g., ``give me
%200 CPU cores, 1 TB memory, and 100 Gbps Internet connectivity'' - but today application developers and datacenter operators must map their resource demand to the granularity of servers or VMs. 
%One can view disaggregation as changing the abstraction offered by the infrastructure from that of a ``pool of servers'' to that of a ``pool of resources''. 
%We believe that the latter offers greater flexibility and will prove to be a more natural and powerful abstraction.
%Finally, one avenue ripe for exploration is that of network management for disaggregated datacenters. 
%Instead of a standalone network management solution, we envisage a unified resource management architecture as a combination of the centralized network controller architectures advocated by work on 4D~\cite{4d} and SDN~\cite{sdn} and the job schedulers found in existing datacenters~\cite{mesos}.
%This tight integration of network and resource scheduling can enable greater flexibility; for example, a scheduler can seamlessly migrate resources to detour congested links (recall that disaggregation decouples resource usage from its physical location). 
%The design of such unified schedulers is an interesting topic for future work.







%Going forward we anticipate work in designing disaggregated datacenters will fall into three categories - low-latency networks, network abstractions, and system architecture. We summarize these areas, discussed earlier in prior work~\cite{hotnets}, below.

%\paragraphb{Low-latency networks}
%As demonstrated in \S\ref{sec:requirements} low latency networks will be a critical part of building  disaggregated datacenters on a large scale. Fortunately this is a hot topic in networking~\cite{lowlatency} and networks will likely achieve the bounds specific in this paper.
%We believe reducing network latency can be achieve in the following directions.

%At physical layer, newer hardware technologies, such as cut-through switching, can reduce overhead of packet transmission in datacenter network. All optimal switches can reduce the network latency by removing queuing delay, which is the most significant contributor to the delay in current datacenter architecture.
%At network layer, using TDMA based network architecture proposed by Vattikonda et al.~\cite{tdma} makes the network latency more deterministic. 
%Orthogonal to that, at transport layer, improved protocol design such as pFabric~\cite{pfabric} and pHost~\cite{phost} can further reduce latency by scheduling flows as quickly as possible. 
%Fourth, at the OS layer, applying techniques such as zero-copy~\cite{netmap} can significantly reduce OS network stack overhead and hence reduces latency.
%Remote Direct Memory Access (RDMA) directly accesses the remote memory by circumventing the operating system of the remote machine.
%At application layer, end-to-end delay can be improved by reducing the distance between the job and data. There is much active research~\cite{endpoint} on data and job placement in map-reduce clusters. Future research could study how to optimally place data when storage resources are disaggregated.

%\paragraphb{Network Architecture}
%It is worth noting that disaggregation effectively blurs the lines between what used to be separate intra- and inter-server networks as observed in \S\ref{sec:workloads}. 
%Basic concepts such as variable-sized packets and best-effort service are common in inter-server networks but not so in intra-server links/networks, causing differences in traffic characteristics. 
%We believe it worth exploring the following aspects of the network architecture.

%The first question we may ask is whether packet still remains as the right abstraction for disaggregated datacenters. 
%We found that most of the intra- and inter-server networks use packets switched network, so we believe packets should still be the correct abstraction. 
%The open questions are whether packet size should be fixed for disaggregated datacenters, and what should be the size of packets. 
%While using larger MTUs amortizes packet header processing overhead, and improves the network throughput, it increases the network latency for small message passing workload.

%A second question is how should reliability being guaranteed. 
%While it is clear that each resource endpoint need an end-to-end reliable transfer abstraction, it is unclear where this functionality should be placed.
%Inter-server link technologies such as Infini-Band guarantees reliability at each link, but Ethernet use end-to-end retransmission for reliability. 

%The last question is related to bandwidth reservation. In Intra-server network, bandwidth is usually reserved for each link (CPU-Memory, CPU-NIC). 
%It would be interesting to explore whether using statistical multiplexing is sufficient after disaggregation these components.




%\paragraphb{System Architecture}

%The cost of hardware and its maintenance has been the most powerful driving force of datacenter evolution.
%We believe that disaggregation could further reduces the cost of datacenter for the following reasons.
%First, the datacenter operator has finer-grained control over resource provisioning. 
%Second, resource disaggregation simplifies management complexity. 
%Lastly, the unified network cuts out a layer of integration by avoiding PCIe-Ethernet-PCIe traverse in current server-to-server communication.
%Although the cost of disaggregation is hard to quantify at this moment, we suspect that
%cost savings might turn out to be one of the strongest motivations for disaggregated datacenters.

%While this paper used a VM abstraction for emulation disaggregated datacenters, it is unclear whether this will remain the right abstraction going forward.
%For example, a job-centric abstraction in which users specify resource needs rather than provisioning VMs could prove to be a more flexible and natural abstraction to express tasks in the future.

%Finally, we envisage that a unified resource management architecture that combines network controller and datacenter resource scheduler gives more flexibility for resource management.
%For example, the resource controller can migrate resources from hotspots to less congested links.
%We leave the design of such resource management architecture for future work.



%\begin{itemize}
%	\item Limitations
%		\begin{itemize}
%			\item Ignores application-level constraints
%				\begin{itemize}
%					\item Flow inter-dependency
%					\item data placement and striping
%					\item Deadlines
%				\end{itemize}				
%			\item Could design applications for DDC
%				\begin{itemize}
%					\item Not the focus of this paper; focus on showing feasibility rather than new designs
%					\item better scheduling mechanisms
%					\item better resource packing
%					\item 
%				\end{itemize}				
%			\item Ignored failures, etc.
%			\item mix of applications
%		\end{itemize}		
%	\item Research Challenges: looking ahead
%		\begin{itemize}
%			\item Designing low-latency networks
%			\item Network abstractions for DDC?
%			\item OS and system architecture
%		\end{itemize}		
%\end{itemize}
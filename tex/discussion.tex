\section{Related Work and Discussion}
\label{sec:discussion}

\subsection{Related Work}
As mentioned earlier, there is growing interest in disaggregation in both industry and research~\cite{hptm, seamicro, fdr, rsa, firebox, sonuma, ddcHwDesign1, ddcHwDesign2, ddcHwDesign3, hotnets}. We build on these efforts --– in particular prior work on understanding how memory might be disaggregated~\cite{ddcHwDesign2, ddcHwDesign3} as a starting point. To date, these efforts consider disaggregation at server or rack scale and, as such, they use specialized interconnects designed for their specific context and do not discuss network support for disaggregation more generally nor consider the possibility of leveraging known datacenter network technologies to enable disaggregation; e.g., the network in SeaMicro’s architecture implements a 3D torus interconnect, which only disaggregates I/O and does not scale beyond the rack.

\subsection{Limitations}
One limitation of our results is our application-agnostic measurement approach. While this models disaggregation at the operating system level, it misses any potential application-specific considerations such as the interdependency of flows and its effect on performance. As a result we did not model potential application-specific optimizations such as custom data placement strategies or prioritizing flows based on the access dependency graph. We also did not study failure modes or running mixes of applications together. Our results therefore focus on demonstrating the feasibility of disaggregation rather than showcasing novel designs. Future work in this regard may focus on improving resource packing or data access scheduling mechanisms.

\subsection{Looking Ahead}
Going forward we anticipate work in designing disaggregated datacenters will fall into three categories - low-latency networks, network abstractions, and system architecture. We summarize these areas, discussed earlier in prior work~\cite{hotnets}, below.

\paragraphb{Low-latency networks}
As demonstrated in \S\ref{sec:requirements} low latency networks will be a critical part of building  disaggregated datacenters on a large scale. Fortunately this is a hot topic in networking~\cite{lowlatency} and networks will likely achieve the bounds specific in this paper.
We believe reducing network latency can be achieve in the following directions.

At physical layer, newer hardware technologies, such as cut-through switching, can reduce overhead of packet trasmission in datacenter network. All optimal switches can reduce the network latency by removing queuing delay, which is the most significant contributor to the delay in current datacenter architecture.
At network layer, using TDMA based network architecture proposed by Vattikonda et al.~\cite{tdma} makes the network latency more deterministic. 
Orthogonal to that, at transport layer, improved protocol design such as pFabric~\cite{pfabric} and pHost~\cite{phost} can further reduce latency by scheduling flows as quickly as possible. 
Fourth, at the OS layer, applying techniques such as zero-copy~\cite{netmap} can significantly reduce OS network stack overhead and hence reduces latency.
Remote Direct Memory Access (RDMA) directly accesses the remote memory by circumventing the operating system of the remote machine.
At application layer, end-to-end delay can be improved by reducing the distance between the job and data. There is much active research~\cite{endpoint} on data and job placement in map-reduce clusters. Future research could study how to optimally place data when storage resources are disaggregated.

\paragraphb{Network Architecture}
It is worth noting that disaggregation effectively blurs the lines between what used to be separate intra- and inter-server networks as observed in \S\ref{sec:workloads}. 
Basic concepts such as variable-sized packets and best-effort service are common in inter-server networks but not so in intra-server links/networks, causing differences in traffic characteristics. 
We believe it worth exploring the following aspects of the network architecture.

The first question we may ask is whether packet still remains as the right abstraction for disaggregated datacenters. 
We found that most of the intra- and inter-server networks use packets swtiched network, so we believe packets should still be the correct abstraction. 
The open questions are whether packet size should be fixed for disaggregated datacenters, and what should be the size of packets. 
While using larger MTUs amortizes packet header processing overhead, and improves the network throughput, it increases the network latency for small message passing workload.

A second question is how should reliability being guaranteed. 
While it is clear that each resource endpoint need an end-to-end reliable transfer abstraction, it is unclear where this functionality should be placed.
Inter-server link technologies such as Infini-Band guarantees reliableility at each link, but Ethernet use end-to-end retransmission for reliability. 

The last question is related to bandwidth reservation. In Intra-server network, bandwidth is usually reserved for each link (CPU-Memory, CPU-NIC). 
It would be interesting to explore whether using statistical multiplexing is sufficient after disaggregation these compoenents.




\paragraphb{System Architecture}
While this paper used a VM abstraction for emulation disaggregated datacenters, it is unclear whether this will remain the right abstraction going forward. For example, a job-centric abstraction in which users specify resource needs rather than provisioning VMs could prove to be a more flexible and natural abstraction to express tasks in the future.




%\begin{itemize}
%	\item Limitations
%		\begin{itemize}
%			\item Ignores application-level constraints
%				\begin{itemize}
%					\item Flow inter-dependency
%					\item data placement and striping
%					\item Deadlines
%				\end{itemize}				
%			\item Could design applications for DDC
%				\begin{itemize}
%					\item Not the focus of this paper; focus on showing feasibility rather than new designs
%					\item better scheduling mechanisms
%					\item better resource packing
%					\item 
%				\end{itemize}				
%			\item Ignored failures, etc.
%			\item mix of applications
%		\end{itemize}		
%	\item Research Challenges: looking ahead
%		\begin{itemize}
%			\item Designing low-latency networks
%			\item Network abstractions for DDC?
%			\item OS and system architecture
%		\end{itemize}		
%\end{itemize}
\begin{abstract}
Traditional datacenters are designed as a collection of servers, each of which tightly couples the resources required for computing tasks. Recent industry trends suggest a paradigm shift to disaggregated datacenter (DDC) architecture that comprises of a pool of resources, each built as a standalone resource blade, and interconnected using a network fabric. In this paper, we use a workload-driven approach to explore three questions: (1) what support would DDC need from the network to maintain application-level performance close to traditional datacenters? (2) how will the network traffic change in DDC? and (3) can existing transport protocols meet the DDC application requirements for these new traffic workloads?
\end{abstract}
